\documentclass{article}
\usepackage[utf8]{inputenc}
\usepackage{geometry}
\geometry{a4paper, margin=1in}

\title{Swing Probability Prediction in Major League Baseball Pitches}
\author{Olubayode Ebenezer}
\date{4/23/2024}

\begin{document}

\maketitle

\section*{Problem Statement}
In baseball analytics, understanding the probability that a pitch will be swung at by a batter is crucial for both game strategy and player performance analysis. Currently, there is a gap in predictive analytics for pitches without a direct description of their outcome. The dataset includes comprehensive pitch data over three seasons, but the third season lacks descriptive outcomes for each pitch. This project aims to fill that gap by predicting the swing probability for each pitch in the third season based on the available data.

\subsection*{Additional Objectives}
Identify key factors for specific pitch types by defining and analyzing middle-middle pitches—pitches within 0.5 feet of the center of the strike zone—and determine which variables most significantly influence the decision to swing at these pitches.

Develop comparative player metrics using the model's insights to establish metrics that compare individual players’ performance to league averages, focusing on their decisions to swing, and providing a deeper analysis of player tendencies and skills.

\section*{Approach}
\begin{enumerate}
    \item Data Analysis and Preparation: Analyzed the dataset from the first two seasons to understand patterns and trends associated with swing decisions. Prepared the data by cleaning and selecting relevant features that influence a batter's decision to swing.
    \item Feature Engineering: Conducted feature selection and engineering using polynomial interactions to build the model for swing probability. Selected features found most informative using feature importance, exploring interactions of features to improve model performance.
    \item Model Development and Tuning: Developed a predictive model using the data from the first two seasons to estimate the likelihood of a swing for each pitch, including some hyperparameter tuning. The model considers various factors such as pitch type, count, pitcher, batter stance, and the physical characteristics of the pitch.
    \item Model Validation and Evaluation: Validated the predictive model using a subset of data (e.g., cross-validation) to ensure accuracy and robustness, evaluated it using the F1 Score.
    \item Prediction and Application: Applied the validated model to the third season's data to predict swing probabilities and evaluated the model’s performance in terms of its practical utility for baseball teams and analysts.
    \item Results Documentation and Presentation: Documented the methodology, model development, validation process, and results in a comprehensive manner.
\end{enumerate}

\section*{Methodology}
\subsection*{Data Loading and Merging}
Load the datasets for Year1 and Year2, and the validation data for Year 3. Merge the Year1 and Year2 data into a single dataframe.

\subsection*{Data Exploration}
\begin{itemize}
    \item Numerical Distribution Plots
    \item Data Quality Reports:
    \begin{itemize}
        \item Observations on Missing Values: Significant findings were noted regarding missing values and most suggested randomness in missing data patterns.
    \end{itemize}
\end{itemize}

\subsection*{Data Cleaning}
\begin{itemize}
    \item For normally distributed data, impute missing values using the mean imputation grouped by related features of the group.
    \item For non-normally distributed data, consider median imputation grouped by related features (e.g., pitch type).
    \item For categorical data, mode imputation.
\end{itemize}

\subsection*{Data Preprocessing}
\begin{itemize}
    \item Dropped rows that were missing.
    \item Converted descriptions into a new SwingProbability I called SwingType category based on logical grouping derived from baseball rules and typical game scenarios.
\end{itemize}

\subsection*{Feature Engineering}
Developed feature interactions to capture complex relationships within the data by creating polynomial and interaction features that reflect strategic elements of baseball pitching and batting.

\section*{Modeling and Results}
\subsection*{Baseline Models}
The two baseline models used are CATBOOST and LGBM Classifier. Utilized Smote for Sampling.
\begin{itemize}
    \item Trained Catboost Modeling using the top best 23 predictive features gotten from Features importance (Feature Selection) that was built using LGBM.
    \item Built the second model using LGBM without the features interactions features generated through Polynomial.
\end{itemize}

\subsection*{Final Model and Submission}
\begin{itemize}
    \item Model Prediction: Computed the mean probabilities of the classes for swing prediction by averaging the probabilities obtained from two different models: LightGBM and CatBoost Classifier.
    \item Probability Array Example: 
    The array mean probabilities holds the averaged class probabilities for each pitch. Each row in the array corresponds to a pitch, and each column corresponds to one of the four classes of swing probability: No Swing, Unlikely Swing, Attempt to Swing (bunt), and Definite Swing.
    \item Use argmax to determine the Predicted Class and Then used the year3 data to predict my Swing Likelihood and I then mapped the No Swing(0) and Unlikely Swing(1) to No Swing in my Swing Probability and Mapped Attempt to Swing (bunt)(2), and Definite Swing(3) as Swing.
\end{itemize}

\section*{Evaluation}
\subsection*{Understanding the Baseball Swing Prediction Model}
\begin{itemize}
    \item Metrics Used: F1 Score, combining precision and recall to evaluate model performance.
    \item Why Use the F1 Score: Ensures both the accuracy of swing predictions and their comprehensiveness, critical in game-changing decisions.
    \item Predictions \& Interpretations: Utilized argmax to select the most likely type of swing based on model probabilities.
    \item Model Reliability: Evaluated through cross-validation to ensure robust performance across different data sets.
\end{itemize}

\section*{Further Analysis}
\subsection*{Understanding Middle-Middle Pitches}
\begin{itemize}
    \item Definition and criteria for middle-middle pitch identification.
    \item Impact of pitch location and speed on swing probability.
\end{itemize}

\subsection*{Metric Methodology: Swing Decision Efficiency (SDE)}
Detailed explanation of the calculation and application of the SDE metric to season 2 data, including the classification of swing decisions and aggregation to determine efficiency.

\subsection*{Results and Implications}
Discussion of top and bottom performers based on SDE scores, highlighting differences in swing decision-making skills among players and its implications for coaching and player development.

\section*{Conclusion}
This is just an overview of my project. And my data are gotten from Miami Marlins Baseball team.

\end{document}
